\documentclass[11pt]{article}
\usepackage[utf8]{inputenc}
\usepackage{amsmath}
\usepackage[a4paper, total={6in, 8in}]{geometry}

\title{Evaluación de Axcel Picado}
\author{Fernando Herrera Valverde}
\date{I semestre 2019}

\begin{document}
\maketitle

\begin{itemize}
\item Hay conceptos muy básicos que podrían quitarse porque una persona que este leyendo sobre el juego de Choquet definitivamente va a saber que 
es un espacio métrico o una sucesión de Cauchy.
\item En el teorema 2.2 la colección debe ser numerable.
\item En el teorema 2.3 la colección debe ser numerable, de hecho este teorema es el mismo que el 2.2, entonces podría solo decir que son formas equivalentes de enunciar el teorema.
\item Página 3, error en la palabra reflexiva y en la palabra existe.
\item Definición 3.2, hay un error en el segundo $\forall$
\item Página 4, error en la pabra así.
\item Definición 4.1, el último $V_i$ debería ser $V_{i-1}$ y una línea se sale de los márgenes
\item En el teorema 4.4 dice que todo subconjunto de un espacio de Baire es Baire, según wikipedia, también hace falta ser abierto.
\item Teorema 4.4, error en la palabra vacío, no dice quién es $A$ ni quien es $B$. No me queda muy claro quienes son las familias que menciona en la primera dirección. ¿Quién es $\mathcal{A}^*$?
\item En la página 7 párrafo 2 está escrito un poco raro. Siguiendo el orden de los jugadores podría ser más fácil de entender.
\item Podría poner la referencia del lema 4.6 y teorema 4.7.
\item En mi opinión el trabajo está muy bien, el formato del documento se ve excelente y la redacción me parece muy clara. El teorema 4.4 me pareció muy díficil de entender, de hecho solo entendí la segunda dirección, tal vez se pueden usar lemas o alguna forma de simplificar esa prueba para que no sea tan larga y con tantos pasos. Lo demás me perece muy bueno.
\end{itemize}
\end{document}
